\chapter{State Of The Art}
\label{ch2_SOA}
\thispagestyle{empty}

\vspace{0.5cm}

\noindent This section will be dedicated to the analysis of the two main areas 
of machine learning around which this thesis revolves: \textit{deep learning} 
and \textit{reinforcement learning}. 
After an introduction of both areas, we 
will describe how the two classes of algorithms can be used together in what is 
called \textit{deep reinforcement learning}.

\section{Deep Learning}
Deep Learning is a branch of machine learning which exploits \textit{Artificial 
Neural Networks} (ANN) with more than one hidden layer to learn an abstract 
representation of the input space \cite{lecun2015deep}. 

Deep learning techniques can be applied to the three main classes of problems 
of machine learning (supervised, semi-supervised, and unsupervised), and have 
been used to achieve state-of-the-art results in a variety of learning tasks.


\section{Reinforcement Learning}

\section{Deep Reinforcement Learning}
Deep Reinforcement Learning is the application of RL's sequential decision 
making techniques and DL's abstract representation capabilities to control 
problems, usually on MDPs where the state representation is too complex to be 
tractable with classic dynamic programming techniques. 



