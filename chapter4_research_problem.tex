\chapter{Fitted Q-Iteration with Deep State-Dynamics Features}
\label{ch3_setup}
\thispagestyle{empty}

\vspace{0.5cm}

As central argument of this thesis, we propose a DRL method which 
combines the feature extraction capabilities of deep CNNs with the quick 
and powerful batch RL approach of FQI. 
Given a high-dimensional state space of pixels representing sequences of 
greyscale frames from an Atari game, we use a deep convolutional autoencoder 
to map the original state space to a compressed \textit{feature space} which 
accounts for both the states and their one-step dynamics (i.e. the transition 
model). This compressed representation is then used to run a \textit{tree-based}
FQI algorithm in batch mode.

In this chapter we give a formal description of the method and its core 
components. Technical details of implementation will be discussed in the next
chapter.

\section{Motivation}
The state-of-the-art DRL methods listed in the previous chapter are able to 
outperform classic RL algorithms in a wide variety of problems, and in some 
cases are the only possible way of dealing with high-dimensional control 
settings like the Atari games. 
However, the approaches cited above tend to be grossly 
\textit{sample-inefficient}, requiring tens of millions of samples collected
on-line to reach optimal performance. Several publications successfully deal 
with this aspect, but nonetheless leave room for improvement (lowering at most
by one order of magnitude the number of samples required).
The method introduced by Lange and Riedmiller (2010) \cite{lange2010deep} is 
similar to ours but their dense architecture predates the more modern 
convolutional approaches in image processing and is less suited for complex
tasks than our AE.

The method that we propose tries to improve both aspects of information content
of the compressed feature space and sample efficiency. We extract general 
features from the environments and try to reach better or equivalent performance
in up to two orders of magnitude less samples than DQN on Atari games.

\section{Problem Formulation}
% TODO: add RFS
The general setting of this problem is typical of DRL problems: we use a deep 
ANN to extract a representation of an environment, and use that representation 
to control an agent with standard RL algorithms. We add an additional step after
the deep feature extraction to further reduce the representation down to the 
essential bits of information required to solve the problem by using the 
\textit{Recursive Feature Selection} (RFS) algorithm \cite{castelletti2011tree}.

In our approach we use a modular architecture with three separate stages for the
training phase, and combine the three stages in an end-to-end fashion during
the control phase. The main components of the algorithm are:
%
\begin{enumerate}
    \item a deep convolutional autoencoder which we use to extract a 
    representation of the environment;
    the purpose of the AE is to map the original, pixel-level state space $S$ of
    the environment into a strongly compressed feature space ${\tilde{S}}$ which
    contains information of both the state space and part of the transition 
    model of the environment;
    \item a \textit{Recursive Feature Selection} (RFS) technique to further 
    reduce the state representation $\tilde{S}$ and keep only the truly 
    informative features extracted by the AE, effectively mapping the extracted 
    state-space to a subspace $\hat{S}$.
    \item a \textit{tree-based} FQI learning algorithm which produces an 
    estimator for the action-value function, with $\hat{S}$ as domain. 
\end{enumerate}
%
The three components are separately trained to produce the following 
transformations respectively:
\begin{itemize}
    \item $ENC: S \rightarrow \tilde{S}$, from the pixel representation to a 
    compressed feature space;
    \item $RFS: \tilde{S} \rightarrow \hat{S}$, from the compressed feature space
    to a minimal subspace with the most informative features;
    \item $\hat{Q}^\pi: \hat{S} \times A \rightarrow \mathbb{R}$, an 
    action-value function on $\hat{S}$.
\end{itemize}

After the training phase, we simply combine the three functions to obtain the 
action-value function $Q^\pi: S \times A \rightarrow \mathbb{R}$ as follows: 
%
\begin{IEEEeqnarray}{rCl}
    %
    Q^\pi(s, a) = \hat{Q}^\pi(RFS(ENC(s)), a) \label{eq:final_output}
    %
\end{IEEEeqnarray}
%
A general description of the process is given in Algorithm \ref{alg:FQI-DSDF}.
%
\begin{algorithm}[h]
    \caption{Fitted Q-Iterations with Deep State Features}
    \label{alg:FQI-DSDF}
    \begin{algorithmic}
	\STATE \textbf{Given}: an arbitrary policy $\pi$;
	\STATE Initialize the encoder $ENC: S \rightarrow \tilde{S}$ arbitrarily;
	\STATE Initialize the decoder $DEC: \tilde{S} \rightarrow S$ arbitrarily;
	\REPEAT 
	    \STATE Collect a set $\mathcal{TS}$ of four-tuples $(s \in S, a \in A, r \in \mathbb{R}, s' \in S)$ using $\pi$;
	    \STATE Train the composition $DEC \circ ENC: S \rightarrow S$ using the first column of $\mathcal{TS}$ as input and target;
	    \STATE Build a set $\mathcal{TS}_F$ of four-tuples $(f \in \tilde{S}, a \in A, r \in \mathbb{R}, f' \in \tilde{S})$ by applying the encoder to the first and last column of $\mathcal{TS}$ s.t. $f = ENC(s)$;
	    \STATE Call the RFS feature selection algorithm on $\mathcal{TS}_F$ to obtain a space reduction $FS: \tilde{S} \rightarrow \hat{S}$;
	    \STATE Build a set $\mathcal{TS}_{\hat{F}}$ of four-tuples $(\hat{f} \in \hat{S}, a \in A, r \in \mathbb{R}, \hat{f'} \in \hat{S})$ by applying $RFS$ to the first and last column of $\mathcal{TS}_F$ s.t. $\hat{f} = RFS(f)$;
	    \STATE Call FQI on $\mathcal{TS}_{\hat{F}}$ to produce $\hat{Q}^\pi: \hat{S} \times A \rightarrow \mathbb{R}$;
	    \STATE Combine $\hat{Q}^\pi$, $RFS$ and $ENC$ to produce $Q^\pi: S \times A \rightarrow \mathbb{R}$:
		\[
		Q^\pi(s, a) = \hat{Q}^\pi(RFS(ENC(s)), a)
		\]
	    \STATE Set $\pi(s) = \underset{a}{\arg\max} Q^{\pi}(s, a)$;
	\UNTIL{stopping condition is met;}
    \end{algorithmic}
\end{algorithm}
%
\section{Atari Games}
% TODO: Brief introduction

\section{Extraction of State Features}
The AE used in our approach consists of two main components, namely an 
\textit{encoder} and a \textit{decoder} (cf. \ref{s:AE}). To the end of 
explicitly representing the encoding purpose of the AE, we keep a separate 
notation of the two modules; we therefore refer to two different CNNs, namely 
$ENC: S \rightarrow \tilde{S}$ that maps the original state space to the 
compressed representation $\tilde{S}$, and $DEC: \tilde{S} \rightarrow S$ which
performs the inverse transformation. The full AE is the composition of the two 
networks $AE: DEC \circ ENC: S \rightarrow S$. Note that the composition is 
differentiable end-to-end, and basically consists in \textit{plugging} the last
layer of the encoder as input to the decoder. 

We train the AE ...

\section{Tree-based Recursive Feature Selection}
% TODO

\section{Tree-based Fitted Q-Iteration}
The last component of our training pipeline is the Fitted Q-Iteration algorithm 
applied to the training set $\mathcal{TS}_{\hat{F}}$ of four-tuples
$(\hat{f} \in \hat{S}, a \in A, r \in \mathbb{R}, \hat{f'} \in \hat{S})$, 
obtained by applying RFS to the first and last column of 
$\mathcal{TS}_{\tilde{F}}$.
For consistency with the feature selection algorithm, we use the Extra-Trees 
learning method as function approximator for the action-value function. 
We train the model to output a multi-dimensional estimate, with one value for 
each action that the agent can take in the environment, and we select the single
output value on $\mathbb{R}$ using the action identifier as index (e.g.\ 
$Q(s, 0)$ will return the first dimension of the model's output). 
The final output of the training procedure is an estimator 
$\hat{Q}^\pi: \hat{S} \times A \rightarrow \mathbb{R}$ which maps the compressed
and reduced state-action space to a real value. 

% TODO: image of the full model AE + RFS + FQI
 
The final step of our algorithm consists in composing the outputs of the three 
modules in a single, end-to-end model which can operate directly on the original
feature space $S$ as defined in \eqref{eq:final_output}.
