\chapter{Technical Details and Implementation}
\label{ch5_arch}
\thispagestyle{empty}

\vspace{0.5cm}

In this chapter we show the implementation details of the architecture
used to perform experiments. We try to provide a complete description of the 
parametrization of the components and of the training procedure to ensure 
reproducibility of the experimental results.

\section{Environment Preprocessing}
The Atari simulator that we used as test bench provides the state observations 
in the form of $3 \times 110 \times 84$ RGB images. When an action is executed
in the environment, the simulator repeats the action for four consecutive 
frames of the game and then provides another observation, effectively lowering
the frame rate from $60$ \textit{FPS} to $15$ \textit{FPS} and making the 
effects of actions more evident. 
We perform a preprocessing operation on the states similar to that performed
by Mnih et al.\ in DQN, in order to include all necessary information about
the environment and its dynamics in the new state representation.
First we convert each RGB observation to a 1D greyscale representation in the
discrete 8-bit interval $[0, 255]$ using the \textit{ITU-R 601-2 luma transform}:
%
\begin{IEEEeqnarray}{rCl}
    %
    L = \frac{299}{1000}R + \frac{587}{1000}G + \frac{114}{1000}B
    %
\end{IEEEeqnarray}
%
where $R$, $G$, $B$ are the 8-bit components of the image. We then normalize 
these values in the $[0, 1]$ interval via \textit{max-scaling} (i.e.\ dividing 
each pixel by $255$).
We also reduce the height of the image by two pixels in order to prevent 
information loss due to a rounding operation performed by the AE.
Finally, we concatenate the preprocessed observation to the last three 
preprocessed frames observed from the environment, effectively blowing up the
state space to a $4 \times 108 \times 84$ vector space. The initial state for
an episode is artificially set as a repetition of the first observation provided
by the simulator. 
% TODO: put an image here to expalin the sampling technique.
    
Moreover, in order to facilitate comparisons and improve stability, we remain 
loyal to the methodology used for DQN and perform the same clipping of the 
reward signal in a $[-1, 1]$ interval. 

\section{Autoencoder}
%
\begin{table}[h]
    \centering
    \begin{tabular}{l c c c c c c} 
	\hline
	Type & Input & Output & \# Filters & Filter & Stride & Activation \\ 
	\hline 
	Conv. & $4 \times 108 \times 84$ & $32 \times 26 \times 20$ & 32 & $8 \times 8$ & $4 \times 4$ & ReLU \\ 
	Conv. & $32 \times 26 \times 20$ & $64 \times 12 \times 9$ & 64 & $4 \times 4$ & $2 \times 2$ & ReLU \\ 
	Conv. & $64 \times 12 \times 9$ & $64 \times 10 \times 7$ & 64 & $3 \times 3$ & $1 \times 1$ & ReLU \\ 
	Conv. & $64 \times 10 \times 7$ & $16 \times 8 \times 5$ & 16 & $3 \times 3$ & $1 \times 1$ & ReLU \\ 
	Flatten	& $16 \times 8 \times 5$ & $640$ & - & - & - & - \\ 
	\hline
	Reshape & $640$ & $16 \times 8 \times 5$ & - & - & - & - \\
	Deconv. & $16 \times 8 \times 5$ & $16 \times 10 \times 7$ & 16 & $3 \times 3$ & $1 \times 1$ & ReLU \\ 
	Deconv. & $16 \times 10 \times 7$ & $64 \times 12 \times 9$ & 64 & $3 \times 3$ & $1 \times 1$ & ReLU \\
	Deconv. & $64 \times 12 \times 9$ & $64 \times 26 \times 20$ & 64 & $4 \times 4$ & $2 \times 2$ & ReLU \\
	Deconv. & $64 \times 26 \times 20$ & $32 \times 108 \times 84$ & 32 & $8 \times 8$ & $4 \times 4$ & ReLU \\
	Deconv. & $32 \times 108 \times 84$ & $4 \times 108 \times 84$ & 4 & $1 \times 1$ & $1 \times 1$ & Sigmoid \\
	\hline
    \end{tabular}
    \caption{Layers of the autoencoder with key parameters}
    \label{t:AE_structure}
\end{table}
%
We structure the AE to take as input the the preprocessed observations from the
environment and predict values on the same vector space.
The first four convolutional layers make up the encoder and perform a 2D 
convolution with the \textit{valid} padding algorithm such that the input of 
each layer (in the format $channels \times height \times width$) is reduced 
automatically across the last two dimensions (height and width) according to the
following formula: 
%
\begin{IEEEeqnarray}{rCl}
    %
    output_i = \lfloor(input_i - filter_i  + stride_i) / stride_i\rfloor
    %
\end{IEEEeqnarray}
%
Since the main purpose of pooling layers is to provide translation invariance to 
the representation of the CNN (meaning that slightly shifted or tilted inputs
are considered the same by the network), here we choose to not use pooling 
layers in order to preserve the precious information regarding the position of
different elements in the games; this same approach was adopted in DQN.
A final \textit{Flatten} layer is added at the end of the encoder to provide a 
1D representation of the feature space, which is reversed before the beginning 
of the decoder. 

The decoder consists of deconvolutional layers with symmetrical filter sizes, 
filter numbers and strides with respect to the encoder. Here the \textit{valid} 
padding algorithm is inversed to expand the representation with this formula:
%
\begin{IEEEeqnarray}{rCl}
    %
    output_i = \lfloor (input_i \cdot stride_i) + filter_i  - stride_i\rfloor
    %
\end{IEEEeqnarray}
% 
A final deconvolutional layer is added at the end to reduce the number of 
channels back to the original four, without changing the width and height of
the frames (i.e.\ using unitary filters and strides).
All layers in the autoencoder use the \textit{Rectified Linear Unit} (ReLU) 
\cite{nair2010rectified, krizhevsky2012imagenet} nonlinearity as activation 
function, except for the last layer which uses \textit{sigmoids} to limit the 
activations values in the same $[0, 1]$ interval of the input.
Details of the AE layers are summarized in Table \ref{t:AE_structure}.

We train the AE with the Adam optimization algorithm \cite{kingma2014adam} 
(see Table \ref{t:adam_params} for details on the hyperparameters) set to 
minimize the \textit{binary crossentropy} loss defined as:
%
\begin{IEEEeqnarray}{rCl}
    %
    L(y, \hat y) = - \frac{1}{N} \sum\limits_{n=1}^{N}[y_n log(\hat y_n) + (1-y_n) log(1-\hat y_n)]
    %
\end{IEEEeqnarray}
%
where $y$ and $\hat y$ are vectors of $N$ target and predicted observations in 
the state space. The dataset used for training is a subset of the dataset 
collected for the whole learning procedure described in Algorithm 
\ref{alg:FQI-DSDF}, namely the first and last elements of the four-tuples
$(s, a, r, s') \in \mathcal{TS}$.
%
\begin{table}[h]
    \centering
    \begin{tabular}{l c} 
	\hline
	Parameter & Value \\ 
	\hline 
	Learning rate &  0.001 \\
	Batch size & 32 \\
	Exponential decay rate ($\beta_1$) & 0.9 \\
	Exponential decay rate ($\beta_2$) & 0.999 \\
	Fuzz factor ($\varepsilon$) & $10^{-8}$ \\
	\hline
    \end{tabular}
    \caption{Optimization hyperparameters for Adam}
    \label{t:adam_params}
\end{table}
%

We prevent \textit{overfitting} of the training set by monitoring the 
performance of the AE on a held-out set of validation samples, and stopping the 
procedure when the validation loss does not improve for five consequent 
training epochs. 

Finally, we modify the loss function to account for the sparsity of the reward
signal of the environment. Atari games produce a positive reward for the agent
only on rather rare events, such as scoring a point in \textit{Pong} or breaking
a brick in \textit{Breakout}, whereas for the majority of the time the agent
collects a null reward. This means that any training set with samples collected
by playing a game will have an unbalance between the transitions in which the 
game is in a \textit{nominal} behavior and those in which the agent collects the
reward. 
To deal with this unbalance, we introduce a weighting factor by which we scale 
the gradient associated to each sample. Since each training sample of the AE
is associated to a transition $(s_i, a_i, r_i, s'_i) \in \mathcal{TS}$, we 
compute the sample weights for $(s_i, s'_i)$ as the inverse probability of 
observing a transition with reward $r_i$ in $\mathcal{TS}$:
%
\begin{IEEEeqnarray}{rCl}
    %
    SW_i = \frac{1}{P((s, a, r_i, s') | \mathcal{TS})} 
    %
\end{IEEEeqnarray}
%

% Why flatten
% Why last layer
% Why sigmoid in last layer
% Sample preprocessing (scale to 0-1, binarize, cut)
% Total number of parameters
% Adam w/ parameters
% Batch size
% Dataset size
% Training accuracy / loss + plots
% Loss: binary crossentropy
% Sample weights
% Reward clipping

\section{Tree-based Recursive Feature Selection}
% I/O (?)
% Parametrization of trees
% Significance

\section{Tree-based Fitted Q-Iteration}
% I/O
% Parametrization of trees




