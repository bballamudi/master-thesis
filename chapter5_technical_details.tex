\chapter{Technical Details and Implementation}
\label{ch5_arch}
\thispagestyle{empty}

\vspace{0.5cm}

In this chapter we show the implementation details of the architecture
used to perform experiments. We try to provide a complete description of the 
parametrization of the components and of the training procedure to ensure 
reproducibility of the experimental results.

\section{Atari Environments}
The \textit{Arcade Learning Environment} (ALE) \cite{bellemare2013arcade} is an 
evaluation platform for RL agents.
ALE offers a programmatic interface to hundreds of game environments for the 
\textit{Atari 2600}, a popular home video game console developed in 1977 with 
more than 500 games available, and is often referred to simply as 
Atari environments (or games). 
We use the implementation of ALE provided by the \textit{Gym 0.8.1} package for 
Python 2.7, developed by OpenAI and maintained as an open source project. 
This implementation provides access to the game state in the form of 
$3 \times 110 \times 84$ RGB frames, produced at an internal frame-rate of $60$ 
frames per second (FPS).
When an action is executed in the environment, the simulator repeats the action 
for four consecutive frames of the game and then provides another observation, 
effectively lowering the frame rate from $60$ FPS to $15$ FPS and making the 
effects of actions more evident. 
We perform a preprocessing operation on the states similar to that performed
by Mnih et al.\ in DQN \cite{mnih2015human}, in order to include all necessary 
information about the environment and its nominal dynamics in the new state 
representation.
First we convert each RGB observation to a single-channel greyscale 
representation in the discrete 8-bit interval $[0, 255]$ using the 
\textit{ITU-R 601-2 luma transform}:
%
\begin{IEEEeqnarray}{rCl}
    %
    L = \frac{299}{1000}R + \frac{587}{1000}G + \frac{114}{1000}B
    %
\end{IEEEeqnarray}
%
where $R$, $G$, $B$ are the 8-bit components of the image. We then normalize 
these values in the $[0, 1]$ interval via \textit{max-scaling} (i.e.\ dividing 
each pixel by $255$).
We also reduce the height of the image by two pixels in order to prevent 
information loss due to a rounding operation performed by the convolution
method used in the AE.
Finally, we concatenate the preprocessed observation to the last three 
preprocessed frames observed from the environment, effectively blowing up the
state space to a $4 \times 108 \times 84$ vector space. The initial state for
an episode is artificially set as a repetition of the first observation provided
by the simulator. 
% TODO: put an image here to expalin the sampling technique.
    
Moreover, in order to facilitate comparisons and improve stability, we remain 
loyal to the methodology used for DQN and perform the same clipping of the 
reward signal in a $[-1, 1]$ interval. 

We add a final tweak to the Gym implementation of ALE in order to fix a 
requirement of some of the environments to perform some specific actions in 
order to start an episode (e.g.\ in \textit{Breakout} it is required that the 
agent takes actions 1, 4 or 5 to start the game). We automatically start each 
episode by randomly selecting one of the initial actions of the game and forcing
the agent to take that action at the beginning of the episode. 

\section{Autoencoder}
%
\begin{table}[h]
    \centering
    \begin{tabular}{l c c c c c c} 
	\hline
	Type & Input & Output & \# Filters & Filter & Stride & Activation \\ 
	\hline 
	Conv. & $4 \times 108 \times 84$ & $32 \times 26 \times 20$ & $32$ & $8 \times 8$ & $4 \times 4$ & ReLU \\ 
	Conv. & $32 \times 26 \times 20$ & $64 \times 12 \times 9$ & $64$ & $4 \times 4$ & $2 \times 2$ & ReLU \\ 
	Conv. & $64 \times 12 \times 9$ & $64 \times 10 \times 7$ & $64$ & $3 \times 3$ & $1 \times 1$ & ReLU \\ 
	Conv. & $64 \times 10 \times 7$ & $16 \times 8 \times 5$ & $16$ & $3 \times 3$ & $1 \times 1$ & ReLU \\ 
	Flatten	& $16 \times 8 \times 5$ & $640$ & - & - & - & - \\ 
	\hline
	Reshape & $640$ & $16 \times 8 \times 5$ & - & - & - & - \\
	Deconv. & $16 \times 8 \times 5$ & $16 \times 10 \times 7$ & $16$ & $3 \times 3$ & $1 \times 1$ & ReLU \\ 
	Deconv. & $16 \times 10 \times 7$ & $64 \times 12 \times 9$ & $64$ & $3 \times 3$ & $1 \times 1$ & ReLU \\
	Deconv. & $64 \times 12 \times 9$ & $64 \times 26 \times 20$ & $64$ & $4 \times 4$ & $2 \times 2$ & ReLU \\
	Deconv. & $64 \times 26 \times 20$ & $32 \times 108 \times 84$ & $32$ & $8 \times 8$ & $4 \times 4$ & ReLU \\
	Deconv. & $32 \times 108 \times 84$ & $4 \times 108 \times 84$ & $4$ & $1 \times 1$ & $1 \times 1$ & Sigmoid \\
	\hline
    \end{tabular}
    \caption{Layers of the autoencoder with key parameters}
    \label{t:AE_structure}
\end{table}
%
We structure the AE to take as input the preprocessed observations from the
environment and predict values on the same vector space.
The first four convolutional layers make up the encoder and perform a 2D 
convolution with the \textit{valid} padding algorithm such that the input of 
each layer (in the format $channels \times height \times width$) is reduced 
automatically across the last two dimensions (height and width) according to the
following formula: 
%
\begin{IEEEeqnarray}{rCl}
    %
    output_i = \lfloor(input_i - filter_i  + stride_i) / stride_i\rfloor
    %
\end{IEEEeqnarray}
%
Since the main purpose of pooling layers is to provide translation invariance to 
the representation of the CNN (meaning that slightly shifted or tilted inputs
are considered the same by the network), here we choose to not use pooling 
layers in order to preserve the precious information regarding the position of
different elements in the games; this same approach was adopted in DQN.
A final \textit{Flatten} layer is added at the end of the encoder to provide a 
1D representation of the feature space, which is reversed before the beginning 
of the decoder. 
The decoder consists of deconvolutional layers with symmetrical filter sizes, 
filter numbers and strides with respect to the encoder. Here the \textit{valid} 
padding algorithm is inversed to expand the representation with this formula:
%
\begin{IEEEeqnarray}{rCl}
    %
    output_i = \lfloor (input_i \cdot stride_i) + filter_i  - stride_i\rfloor
    %
\end{IEEEeqnarray}
% 
A final deconvolutional layer is added at the end to reduce the number of 
channels back to the original four, without changing the width and height of
the frames (i.e.\ using unitary filters and strides).
All layers in the autoencoder use the \textit{Rectified Linear Unit} (ReLU) 
\cite{nair2010rectified, krizhevsky2012imagenet} nonlinearity as activation 
function, except for the last layer which uses \textit{sigmoids} to limit the 
activations values in the same $[0, 1]$ interval of the input.
Details of the AE layers are summarized in Table \ref{t:AE_structure}.

We train the AE with the Adam optimization algorithm \cite{kingma2014adam} 
(see Table \ref{t:adam_params} for details on the hyperparameters) set to 
minimize the \textit{binary crossentropy} loss defined as:
%
\begin{IEEEeqnarray}{rCl}
    %
    L(y, \hat y) = - \frac{1}{N} \sum\limits_{n=1}^{N}[y_n log(\hat y_n) + (1-y_n) log(1-\hat y_n)]
    %
\end{IEEEeqnarray}
%
where $y$ and $\hat y$ are vectors of $N$ target and predicted observations in 
the state space. The dataset used for training is a subset of the dataset 
collected for the whole learning procedure described in Algorithm 
\ref{alg:FQI-DSDF}, namely the first and last elements of the four-tuples
$(s, a, r, s') \in \mathcal{TS}$.
%
\begin{table}[h]
    \centering
    \begin{tabular}{l c} 
	\hline
	Parameter & Value \\ 
	\hline 
	Learning rate &  $0.001$ \\
	Batch size & $32$ \\
	Exponential decay rate ($\beta_1$) & $0.9$ \\
	Exponential decay rate ($\beta_2$) & $0.999$ \\
	Fuzz factor ($\varepsilon$) & $10^{-8}$ \\
	\hline
    \end{tabular}
    \caption{Optimization hyperparameters for Adam}
    \label{t:adam_params}
\end{table}
%
We prevent \textit{overfitting} of the training set by monitoring the 
performance of the AE on a held-out set of validation samples, and stopping the 
procedure when the validation loss does not improve for five consequent 
training epochs. 

Finally, we modify the loss function to account for the sparsity of the reward
signal of the environment. Atari games produce a positive reward for the agent
only on rather rare events, such as scoring a point in \textit{Pong} or breaking
a brick in \textit{Breakout}, whereas for the majority of the time the agent
collects a null reward. This means that any training set with samples collected
by playing a game will have an unbalance between the transitions in which the 
game is in a \textit{nominal} behavior and those in which the agent collects the
reward. 
To deal with this unbalance, we introduce a weighting factor by which we scale 
the gradient associated to each sample. Since each training sample of the AE
is associated to a transition $(s_i, a_i, r_i, s'_i) \in \mathcal{TS}$, we 
compute the sample weights for $(s_i, s'_i)$ as the inverse probability of 
observing a transition with reward $r_i$ in $\mathcal{TS}$:
%
\begin{IEEEeqnarray}{rCl}
    %
    SW_i = \frac{1}{P((s, a, r_i, s') | \mathcal{TS})} 
    %
\end{IEEEeqnarray}
%

% Why flatten
% Why last layer
% Why sigmoid in last layer
% Sample preprocessing (scale to 0-1, binarize, cut)
% Total number of parameters
% Adam w/ parameters
% Batch size
% Dataset size
% Training accuracy / loss + plots
% Loss: binary crossentropy
% Sample weights
% Reward clipping

\section{Tree-based Recursive Feature Selection}
% I/O (?)
% Parametrization of trees
% Significance
We use the RFS algorithm to reduce the state space representation computed by 
the AE down to the feature space $\hat{S}$.
We base both the feature ranking method $FR$ and the model $\hat{f}$ used for
computing the descriptiveness of the features on the Extra-Trees algorithm for
supervised learning (cf.\ Section \ref{s:extra-trees}).

The feature ranking approach with Extra-Trees is based on the idea of scoring
each input feature by estimating the variance reduction produced anytime that 
the feature is selected during the tree building process. The ranking score is
computed as the percentage of variance reduction achieved by each feature over 
the $M$ trees built by the algorithm.
% TODO: maybe put the FR formula (6) from the paper here?
At the same time, Extra-Trees is a sufficiently powerful and computationally 
efficient supervised learning algorithm to use as $\hat{f}$.
Note that in general $FR$ and $\hat{f}$ could be different algorithms with 
different parametrization, but Castelletti et al.\ use the same regressor for 
both tasks (as does the code implementation of RFS and IFS that we used for 
experiments), and we therefore complied with this choice.
The parametrization of Extra-Trees for $FR$ and $\hat{f}$ is reported in Tables
\ref{t:RFS_tree_params}\footnote{We use two different values for the minimum 
number of samples required to split an internal node or a leaf.} and 
\ref{t:RFS_extra_params}, respectively for the parameters of the base trees used
to build the ensemble and the Extra-Trees algorithm itself (cf.\ Section 
\ref{s:extra-trees}).
%
\begin{table}[h]	
    \centering
    \begin{tabular}{l c} 
	\hline
	Parameter & Value \\ 
	\hline 
	Scoring method &  Variance reduction \\
	Max tree depth & None \\
	$n_{min}$ (node) & $5$\\
	$n_{min}$ (leaf) & $2$ \\
	\hline
    \end{tabular}
    \caption{Parameters for base estimators in Extra-Trees (RFS)}
    \label{t:RFS_tree_params}
\end{table}
%
%
\begin{table}[h]
    \centering
    \begin{tabular}{l c} 
	\hline
	Parameter & Value \\ 
	\hline 
	$M$ (Number of base estimators) & $50$ \\
	$K$ (Number of randomly selected attributes) &  All available attributes \\
	\hline
    \end{tabular}
    \caption{Parameters for Extra-Trees (RFS)}
    \label{t:RFS_extra_params}
\end{table}
%
We use the $R^2$ metric defined in Equation \eqref{e:R2} to compute the ability
of the selected features to describe the target in $K$-fold cross validation 
over the training set $\mathcal{D}$. We compute a confidence interval over the 
scores of the validation predictions as:
\begin{IEEEeqnarray}{rCl}
    %
    CI = \sqrt{\frac{1}{|\mathcal{D}|} \sum\limits_{i = 1}^{|\mathcal{D}|} (R^2)^2 \cdot (\frac{1}{|\mathcal{D}|} \sum\limits_{i = 1}^{|\mathcal{D}|} (R^2))^2}
    %
\end{IEEEeqnarray}
and we adapt the stopping condition to RFS by setting $\epsilon = CI + CI_{old}$ 
(where the suffix $old$ has the same meaning as in Algorithm \ref{alg:IFS}) so
that the condition becomes $R^2 - CI < R^2_{old} - CI_{old}$.
We also multiply $\epsilon$ by a \textit{significance} factor $s$ in order to
control the amount of variance that a feature must explain in order to be added 
to the selection: higher values of $s$ means that the selection will yield a 
smaller subset composed exclusively of very informative features (even if the 
overall amount of variance explained is not necessarily the whole possible 
amount).
The hyperparameters used for RFS are reported in Table \ref{t:RFS}.
%
\begin{table}[h]
    \centering
    \begin{tabular}{l c} 
	\hline
	Parameter & Value \\ 
	\hline 
	$K$ (for $K$-fold cross-validation) & $3$ \\
	$s$ (significance) &  0.5 \\
	\hline
    \end{tabular}
    \caption{Parameters for RFS}
    \label{t:RFS}
\end{table}
%

\section{Tree-based Fitted Q-Iteration}
We use the FQI algorithm to learn an approximation of the $Q$ function form the
compressed and reduced state space extracted by the previous two modules. 
For consistency with the feature selection algorithm, we use the Extra-Trees 
learning method as function approximator for the action-value function.
The model is trained to map the 1D compressed feature space $\hat{S}$ to the 
$|A|$-dimensional action-value space $\mathbb{R}^{|A|}$, and we use the action 
identifiers to select the single output value of our approximated $\hat{Q}$ 
function. 
The parametrization of the decision trees built for the ensemble is reported in
Table \ref{t:FQI_tree_params} and the parametrization specific to Extra-Trees 
is reported in Table \ref{t:FQI_extra_params} (cf.\ Section \ref{s:extra-trees}).
%
\begin{table}[h]	
    \centering
    \begin{tabular}{l c} 
	\hline
	Parameter & Value \\ 
	\hline 
	Scoring method &  Variance reduction \\
	Max tree depth & None \\
	$n_{min}$ (node) & $2$\\
	$n_{min}$ (leaf) & $1$ \\
	\hline
    \end{tabular}
    \caption{Parameters for base estimators in Extra-Trees (FQI)}
    \label{t:FQI_tree_params}
\end{table}
%
%
\begin{table}[h]
    \centering
    \begin{tabular}{l c} 
	\hline
	Parameter & Value \\ 
	\hline 
	$M$ (Number of base estimators) & $100$ \\
	$K$ (Number of randomly selected attributes) &  All available attributes \\
	\hline
    \end{tabular}
    \caption{Parameters for Extra-Trees (FQI)}
    \label{t:FQI_extra_params}
\end{table}
%

Since the FQI procedure introduces a small bias to the action-value estimate at 
each iteration (due to approximation error and a similar over-estimation issue 
to that described in Section \ref{SOA:value} for Q-learning), we implement an 
\textit{early stopping} procedure based on the evaluation of the agent's 
performance under an $\varepsilon$-greedy policy (with $\varepsilon = 0.05$) on 
the current partial approximation $\hat{Q}_i$, where $i$ is the number of FQI 
steps occurred so far. If the agent's performance does not improve for five 
consecutive iterations, we stop the training and produce the best performing 
estimation as output to the training phase.
The evaluation procedure is executed after each fit of the Extra-Trees
algorithm, by composing all the modules in the pipeline to obtain the full $Q$ 
approximation defined in Equation \eqref{eq:final_output}. 
The performance is evaluated by running the policy for five episodes and 
averaging the clipped cumulative return of each evaluation episode. 
We also consider the average number of steps per episode as indicator of the 
agent performance for human assessment, but we never use it as part of the 
algorithm. 

Finally, we consider a discount factor $\gamma = 0.99$ for the MDP in order to 
give importance to rewards in a sufficiently large time frame. 

\section{Evaluation}
An evaluation step is run after each training step of the full procedure.
Similarly to what we do to evaluate the agent's performance during the training
of FQI, here we use an $\varepsilon$-greedy policy with $\varepsilon = 0.05$ 
based on the full $Q$ composition of Equation \eqref{eq:final_output}.
The reason for using a non-zero exploration rate during evaluation is that ALE
provides a fully deterministic initial state for each episode, and by using a 
deterministic policy we would always observe the same trajectories (thus leading
to overfitting to the best policy from the initial state, rather than a generic
good policy from any state). Using a non-zero exploration rate allows us to 
assess the agent's capability of playing effectively in any state of the game, 
and of correcting its behavior after an (albeit artificial) mistake.

We let the agent experience $N$ separate episodes under the $\varepsilon$-greedy 
policy, and for each episode we consider the cumulative clipped return and the
number of steps occurred. The mean and variance of these two metrics across the 
$N$ episodes provide us with insights on the agent's performance: a high mean 
return obviously means that the algorithm has produced a good policy, but at the
same time a low variance in the number of steps could indicate that the agent is 
stuck in some trivial policy (e.g. take always the same action) which causes the
episodes to be essentially identical, even accounting for the non-zero 
exploration rate. The latter aspect, while not negative in general, can help in 
the initial steps of experiments to detect potential problems in the 
implementation.

Note that here too, similarly to what done for the collection of the training 
sets, we force-start each episode by randomly selecting one of the initial 
actions of the game. 

Evaluation parameters are summarized in Table \ref{t:eval}.

%
\begin{table}[h]
    \centering
    \begin{tabular}{l c} 
	\hline
	Parameter & Value \\ 
	\hline 
	Exploration rate $\varepsilon$ & $0.05$ \\
	$N$ &  $10$ \\
	\hline
    \end{tabular}
    \caption{Parameters for evaluation}
    \label{t:eval}
\end{table}
%

