\chapter{Background}
\label{ch1_intro}
\thispagestyle{empty}

\vspace{0.5cm}

\noindent Introduce the section

\section{Deep Learning}
Deep Learning is a branch of machine learning which exploits \textit{Artificial 
Neural Networks} (ANN) with more than one hidden layer to learn an abstract 
representation of the input space \cite{lecun2015deep}. 

Deep learning techniques can be applied to the three main classes of problems 
of machine learning (supervised, semi-supervised, and unsupervised), and have 
been used to achieve state-of-the-art results in a variety of learning tasks.

\subsection{Artificial Neural Networks}
Feed-forward Artificial Neural Networks (ANN) are function approximators 
inspired by the connected structure of neurons and synapses in the animal brain.

\subsection{Convolutional Neural Networks}
Convolutional Neural Networks (CNN) are a type of ANN inspired by the visual 
cortex in animal brains.
CNNs exploit spatially-local correlations in the neurons of adjacent 
layers through the use of a \textit{receptive field}, a set of weights which is 
used to transform a local subset of the input neurons of a layer.

% shared weights
% put some images here

\subsection{Autoencoders}
Autoencoders are a type of ANN which are used to learn a sparse and compressed 
representation of the input space, by sequentially compressing and 
reconstructing the inputs under some sparsity constraint (typically by 
minimizing the $L1$ norm of the activations in the innermost hidden layers).


\section{Reinforcement Learning}

\subsection{Markov Decision Processes}
Markov Decision Processes (MDP) are discrete-time, stochastic control 
processes, that can be used to describe the interaction of an \textit{agent} 
with an \textit{environment}. \\

Formally, MDPs are defined as 7-tuples $(S, S^{t}, A, P, R, \gamma, \mu)$, 
where:
\begin{itemize}
    %
    \item $S$ is the set of observable states of the environment. \\
    When the set observable states coincides with the true set of states of the 
    environment, the MDP is said to be \textit{fully observable}. We will only deal 
    with fully observable MDPs without considering the case of \textit{partially 
    observable} MDPs.

    \item $S^{T} \subseteq S$ is the set of \textit{terminal states} of the 
    environment, meaning those states in which the interaction between the agent and 
    the environment ends. A sequence of states observed by an agent during an 
    interaction with the environment and ending in a terminal state is usually 
    called an \textit{episode}.
 
    \item $A$ is the set of actions that the agent can execute in the environment.
 
    \item $P: S \times A \times S \rightarrow [0,1]$ is a \textit{state transition 
    function} which, given two states $s, s' \in S$ and an action $a \in A$, 
    represents the probability of the agent going to state $s'$ by executing $a$ in 
    $s$. \\
 
    \item $R: S \times A \rightarrow \mathbb{R}$ is a \textit{reward function} 
    which represents the reward that the agent collects by executing an action in 
    a state. 
    
    \item $\gamma \in (0,1)$ is a \textit{discount factor} with which the rewards 
    collected by the agent are diminished at each step, and can be interpreted as 
    the agent's interest for rewards further in time rather than immediate.
    
    \item $\mu: S \rightarrow [0, 1]$ is a probability distribution over $S$ which 
    models the probability of starting the exploration of the environment in a 
    given state.
    %
\end{itemize}

Episodes are usually represented as sequences of tuples \[[(s_0, a_0, r_0, s_1), ..., (s_{n-1}, a_{n-1}, r_{n-1}, s_n)]\]
called \textit{trajectories}, where $(s_i, a_i, r_i, s_{i-1})$ represent a
transition of the agent to state $s_{i+1}$ by taking action $a_i$ in $s_i$ and 
collecting a reward $r_i$, and $s_n \in S^T$. \\
In Markov Decision Processes, the modeled environment must satisfy the 
\textit{Markov property}, meaning that the reward and transition functions of 
the environment must only depend on the current state and action, rather than 
the past state-action trajectory of the agent. \\
In other words, an environment is said to satisfy the Markov property when its 
one-step dynamics allow to predict the next state and reward given only the 
current state and action.

\subsubsection{Policy}
The behavior of the agent in an MDP can be defined as a probability 
distribution $\pi: S \times A \rightarrow [0,1]$ called a \textit{policy}, 
which given $s \in S, a \in A$, represents the probability of selecting $a$ as 
next action from $s$.
An agent which uses this probability distribution to select its next action 
when in a given state is said to be \textit{following} the policy.


\subsubsection{Value Functions}
Starting from the concept of policy, we can now introduce a function that 
evaluates how good it is for an agent following a policy $\pi$ to be in a given 
state. This evaluation is expressed in terms of the expected return, i.e.
the expected discounted sum of future rewards collected by an agent starting 
from a state while following $\pi$, and the function that computes it is 
called the \textit{state-value function for policy $\pi$}.\\

Formally, the state-value function associated to a policy $\pi$ is a function 
$V^{\pi}: S \rightarrow \mathbb{R}$ defined as:

\begin{IEEEeqnarray}{rCl}
    %
    V^{\pi}(s) & = & E_\pi[R_t | s_t = s] \\
    & = & E_\pi[\sum\limits_{k = 0}^{\infty} \gamma^k r_{t+k+1} | s_t = s]
    %
\end{IEEEeqnarray}

where $E_\pi[\cdot]$ is the expected value given that the agent follows 
policy $\pi$, and $t$ is any time step of an episode of $n$ steps $[s_1, 
..., s_t, ..., s_n]$ where $s_t \in S, \forall t = 1, ..., n$. \\

Similarly, we can also introduce a function that evaluates the goodness of 
taking a specific action in a given state, namely the expected reward obtained 
by taking an action $a \in A$ in a state $s \in S$ and then following policy $\pi$. 
We call this function the \textit{action-value function for policy $\pi$} 
denoted $Q^{\pi}: S \times A \rightarrow \mathbb{R}$, and defined as: 

\begin{IEEEeqnarray}{rCl}
    %
    Q^{\pi}(s, a) & = & E_\pi[R_t | s_t = s, a_t = a] \\
    & = & E_\pi[\sum\limits_{k = 0}^{\infty} \gamma^k r_{t+k+1} | s_t = s, a_t = a]
    %
\end{IEEEeqnarray}

The majority of reinforcement learning algorithms is based on computing (or 
estimating) value functions to then control the behavior of the agent in order 
to maximize the expected reward collected during episodes. \\

We also note a fundamental property of value functions, which satisfy particular 
recursive relationships like the following \textit{Bellman equation for $V^{\pi}$}:

\begin{IEEEeqnarray}{rCl}
    %
    V^{\pi}(s) & = & E_\pi[R_t | s_t = s] \nonumber\\
    & = & E_\pi[\sum\limits_{k = 0}^{\infty} \gamma^k r_{t+k+1} | s_t = s] \nonumber\\
    & = & E_\pi[r_{t+1} + \gamma \sum\limits_{k=0}^{\infty} \gamma^k r_{t+k+2} | s_t = s] \\
    & = & \sum\limits_{a \in A} \pi(s, a) \sum\limits_{s' \in S} P(s, a, s')[R(s, a) \>+ \nonumber\\
    && +\> \gamma E_\pi[\sum\limits_{k=0}^{\infty} \gamma^k r_{t+k+2} | s_{t+1} = s]] \\
    & = & \sum\limits_{a \in A} \pi(s, a) \sum\limits_{s' \in S} P(s, a, s')[R(s, a) + \gamma V^{\pi}(s')]
    %
\end{IEEEeqnarray}

Intuitively, this relation decomposes the state-value function as the sum of the
immediate reward collected from a state $s$ to a successor state $s'$, and the 
value of $s'$ itself; by considering the transition model of the MDP and the 
policy being followed, we see that Bellman equation simply averages the expected
return over all the possible $(s, a, r, s')$ transitions, by taking into account 
the probability that these transitions occur. 

\subsection{Optimal Value Functions}
In general terms, \textit{to solve} a reinforcement learning task is to identify
a policy that yields a sufficiently high expected return. In the case of MDPs 
with finite state and actions sets 
\footnote{This specification is only required for formality, but is not 
investigated further in this work. Refer to SUTTON, BARTO for more details on 
the subjet of non-finite MDPs.}, it is possible to define the concept of 
\textit{optimal policy} as the policy which maximizes the expected return 
collected by the agent in an episode. \\
We start by noticing that state-value functions define a partial ordering over 
policies as follows: 

\[
    \pi \ge \pi' \iff V^{\pi}(s) \ge V^{\pi'}(s), \forall s \in S
\]

From this, the \textit{optimal policy $\pi^*$} of an MDP is a policy which is
better or equal than all other policies in the policy space. \\
The state-value function associated to $\pi^*$ is called the 
\textit{optimal state-value function}, denoted $V^*$ and defined as:

\[
    V^*(s) = \max_{\pi} V^\pi(s), \forall s \in S
\]

As we did when introducing the value functions, given an optimal policy for the 
MDP it is also possible to define the \textit{optimal action-value function} 
denoted $Q^*$ (the second equivalence in this definition highlights the relation
between $Q^*$ and $V^*$):

\begin{IEEEeqnarray}{rCl}
    Q^*(s, a) & = & \max_{\pi} Q^\pi(s, a) \\
    & = & E[r_{t+1} + \gamma V^*(s_{t+1} | s_t = s, a_t = a]
\end{IEEEeqnarray}

%% SEE PAGE 224 of Sutton, Barto and ask Restelli WTF is going on with the definition of V*





\subsection{Value-based optimization}

\subsubsection{SARSA}

\subsubsection{Q-learning}

\subsubsection{Fitted Q-Iteration}


\section{Deep Reinforcement Learning}

\subsection{Deep Q-Learning}
\subsection{Asynchronous Advantage Actor Critic}


